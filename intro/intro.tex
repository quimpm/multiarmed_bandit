\begin{document}
		% utilitzant el problema de restless bandit ensenyar que relaxar un problema és una bona heurística del problema principal
		% Algorithms to live by
		% Optimal Stopping, Explore/Explode tradeoff
		% El nostre problema \textit{restless bandit}
		% 
		Aquest curs, per una iniciativa de grup, es va començar a llegir diferents llibres relacionats amb l'enginyeria informàtica, com 
		\textit{Category Theory for Programmers} de Bartosz Milewski, \textit{The Art of Computer Programming}  de D. Knuth, 
		\textit{Clean Code} de Robert C. Martin \cite{catteo,knuth,clean}...\\
		\\
		Un d'aquests llibres, \textit{Algorithms to live by} \cite{alg2liveby}
		ha obert debats sobre diferents algorismes que desconeixíem i un d'aquests es va centrar en un tema concret del llibre que tractava el problema \textit{explore-exploit} i d'una modelització d'aquest anomenat \textit{k-armed bandits}. En el llibre, es comenta per sobre el que implica i quines possibles solucions té, així com també es parla d'una variació més complicada d'aquest problema, el  \textit{restless bandit}. La perspectiva que ens interessa d'aquest problema és el tipus de solució a la que s'arriba sabent la complexitat que té, que com es veurà, s'aconsegueix amb amb la relaxació del mateix.  % TODO fer-ho millor
\end{document}