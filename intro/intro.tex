\begin{document}
		% utilitzant el problema de restless bandit ensenyar que relaxar un problema és una bona heurística del problema principal
		% Algorithms to live by
		% Optimal Stopping, Explore/Explode tradeoff
		% El nostre problema \textit{restless bandit}
		% 
		Aquest curs, per una iniciativa de grup, hem començat a llegir diferents llibres relacionats amb l'enginyeria informàtica, com 
		\textit{Category Theory for Programmers} de Bartosz Milewski, \textit{The Art of Computer Programming}  de D. Knuth, llibres
		de \textit{Clean series} de Robert C. Martin...\\
		\\
		Dins d'aquests llibres, \textit{Algorithms to live by} [] % TODO posa la referència.
		ha obert debats sobre diferents algorismes que desconeixíem. Dins d'aquests, es comenta el problema de
		\textit{k-armed bandits}, que implica i quina solució té. També es comenta superficialment \textit{restless bandit},
		que és una variació del problema més difícil. La perspectiva que ens interessa d'aquest és el tipus de solució
		que es dóna sabent la complexitat que té, que és donar la relaxació del problema.  % TODO fer-ho millor
\end{document}