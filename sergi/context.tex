\begin{document}
% Explore Exploit
El teu estomac gruny. Vas al restaurant Italià que t'encanta, 
o proves el nou Thailandes que acaba d'obrir? Hi vas amb el 
teu millor amic o amb una persona que no coneixes tant i que 
vols coneixer millor? Masa dificil, millor quedar-se a casa. 
Cuines aquella recepta que t'encanta, o optes per buscar-ne 
alguna de nova per Internet? Saps que, potser millor demanar
una pizza a domicili. Demanes la teva preferida o preguntes 
per les especials? Tant dubtar entre una o l'altra te n'hauràs
cansat abans de poguer fer la primera mossegada.
\\\\
Cada dia ens veiem forçats a fer decisions entre dos opcions,
que difereixen en dos dimensions: Ens quedem amb les nostres
coses preferides, o n'explorem de noves? Intuitivament podem
pensar que la vida és un balanç entre les dues, pero la pregunta
és: Quin és el balanç?
\\\\
Molts matemàtics i informàtics han estat treballant en aquest
balanç des de fa més de 50 anys donant-li el nom de explore/exploit
tradeoff.
\\\\
En computació, la tensió entre explorar o explotar pren la seva 
forma més concreta an l'escenari anomenat multi-armed bandit, o
k-armed bandit. Aquest nom li és donat degut a que es la forma
coloquial de referir-se a les màquines escura-butxaques.
% K-armed bandit
Imagina entrar a un cassino ple de màquines escura-butxaques,
cada una amb les seves possibilitats de fer una tirada guanyadora.
Naturalment, s'està interessat en maximitzar els guanys. Està clar
que hi haura una fase d'exploració en la qual testegarem les màquines,
i una altra d'explotació tirant d'aquelles que creiem que són més 
beneficioses.\\
La primera passa cap a la solució va ser l'algorisme Win-Stay, Lose-Shift,
proposat per Herbert Robins. Aquest consisteix en triar a una màquina 
aleatoria, mentres s'obtingui profit jugant en aquella màquina, es continua
jugant en la mateixa i, si després d'una certa tirada la màquina deixa de ser 
profitosa, es canvia a una altra. Aquesta tot i estar lluny d'una solució optima,
es va demostrar que els resultats eren millors que els de la pura sort.\\
No va ser fins al 1970 que John Gittins va trobar una solució optima que 
resolia el problema. Gittins va enfocar el problema en termes de maximitzar
els guanys per un futur que és interminable pero amb 'descontes'. 
Fent així la asumsió de que el valor assignat als guanys decreixia 
geométricament. Per exemple, es creu que hi ha un 1\% de probabilitats de 
ser atropellat per un autobus un día, aleshores s'ha de valorar el sopar del
seguent día un 99\% del valor del d'aquesta nit, només perque l'endemà potser mai
s'arriba a sopar. D'aquesta forma, va arribar a la conclusió de que cada màquina 
de la qual en sabem una mica o res, té un nombre que ens indica la probabilitat
de guany que ens farà decidir si tornar a jugar en ella o no. Aquest
nombre és conegut com l'index de Gittins.\\\\
% Introduccio a restless bandit
Una variació d'aquest problema(multi-armed bandit), és que cada una de 
les màquines escura-butxaques es comporta com una màquina Markov. Es a dir,
cada cop que una màquina en particular és jugada, l'estat d'aquesta canvia
a un nou escollit dacord a l'evolució de probabilitats dels estats d'aquesta màquina de Markov.
I si a la variació anterior se li aplica, que l'estat de les màquines no jugades 
pot evolucionar al llarg del temps, apareix el problema del restless bandit.  
% Exemples reals de restless bandit: medicament per un virus es 
\end{document}
