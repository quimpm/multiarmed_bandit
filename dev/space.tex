\begin{document}
	% PSPACE-hard problem. Que significa, i que fa.
	% Les solucions del problema es faran a base d'heuristiques ja que calcular-ho seria massa dificultos
	Dins de les complexitats que poden tenir els problemes, la complexitat de PSPACE compleix:
	\[
	{NP} \subseteq PSPACE \subseteq {EXP} % - time
	\]
	De la mateixa manera que no es sap si $P = NP$ tampoc es sap si $NP = PSPACE$. La demostració que NP es contingut dins de
	PSPACE es realitzar per reducció a l'absurd:\\
	\\
	 Sigui M una màquina de Turing NP, és a dir, donada una instància d'un problema ens diu en un temps 
	polinomial si aquest pertany al problema. Si l'espai per a desenvolupar l'algorisme fos més gran a polinòmic, llavors forçosament
	per a llegir o escriure aquesta informació es necessitaria aquest temps, pel que arriba a l'absurd amb la definició de la màquina M. % TODO: posar-ho més explicat
	\\
	\\
	En aquesta complexitat tenim el problema de \textit{Restless bandit}, on l'explicació de per què és així es va fer al 1999, en 
	l'article [].% TODO posar la referència. 
	En aquest s'explica un problema de xarxes que es demostra ser exponencial. Al mateix temps, un problema relaxat d'aquest
	es demostra ser PSPACE-complet, és a dir, tots els problemes de PSPACE poden ser reduïts a aquest problema i aquest pot
	ser reduït a tots els problemes de PSPACE. Finalment es dóna una fórmula de cost del problema de \textit{restless bandit} i
	es redueix el problema relaxat a aquest, demostrant que és PSPACE-hard (tots els problemes de PSPACE són
	reduïbles a aquest). %TODO: tot això és molt dens, no sé fins a quin punt està bé.
	
\end{document}