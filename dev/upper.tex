\begin{document}
% La versió relaxada del problema i la seva solució en Index de Gittins i Upper confidence bound
\subsection{Upper confidence bound}
Que el problema es mostri com a PSPACE-hard significa que el càlcul del mínim és, si no es demostra la igualtat entre
les diferents complexitats, tan dificultós que deixa de ser pràctic intentar-ho. Per sort, el problema relaxat a un
\textit{k-armed bandits} dona una bona heurística al problema principal. Aquest ja hem dit que és decidible amb una complexitat P
per l'índex de Gittins, però el problema recau en el cost de calcular aquest índex, ja que tot i que sigui polinomial en el temps
el cost és bastant elevat, pel que s'utilitzen altres algoritmes que tot i no ser perfectes es queden molt propers a l'òptim del
problema relaxat. Entre aquests algorismes trobem el d'\textit{Upper confidence bound}. \\
\\
L'algorisme UCB fa les seleccions de quina màquina jugar basant-se en l'optimisme. És a dir, centrant-se en el millor que podria tindre dur a terme una acció, donada la evidència obtinguda fins al moment.
Seguint aquesta estratègia, una heurística possible és fer la selecció de la màquina seguint la següent formula:
\[
A_n = argmax_a(Q_n(a) + c\sqrt{\frac{\log (n)}{k_n(a)}})
\]
On $Q_n(a)$ és el valor mitjà actual d'obtenir una recompensa realitzant l'acció de jugar en una màquina escura-butxaques $a$.
El valor sota l'arrel és el logaritme del nombre de màquines a les quals hem jugat, dividit per $k_n$,
que és el nombre de tirades que hem fet en la màquina $a$. I finalment $c$, que és una constant a escollir.\\
Per calcular $Q_n(a)$ només es mantindran en memoria dos valors per cada acció, la
mitja actual ($m_n$) i el nombre seleccions que s'han fet per arribar a aquesta acció $k_n$: %TODO ficar formula
\[
m_{n+1} = m_n + \frac{R_n - m_n}{k_n}
\]
On $R_n$ és la recompensa que obtenim de realitzar aquesta acció.
Aleshores, cada cop que fem una jugada s'avaluaran totes les màquines amb les heurístiques que s'han definit i es jugarà en aquella que obtingui el valor màxim.

\end{document}